\documentclass[10.5pt]{jsarticle}
\pagestyle{plain}
\usepackage{amssymb, amsmath}
\usepackage[margin=20mm]{geometry}
\usepackage{caption}
\usepackage{enumerate}
\renewcommand{\baselinestretch}{1.2}
\renewcommand{\thesection}{\arabic{section}.}
\renewcommand{\labelitemi}{$\blacktriangleright$}
\renewcommand{\labelenumi}{\bf \arabic{section}-\arabic{enumi}.}

\begin{document}

\part*{\bf \Huge inochi Gakusei Innovators' Program (i-GIP) 2022 中高生 募集要項}
\vspace{5mm}

\section{テーマ}
\begin{tabular}{ll}
KANSAI, \ KANTO, \ HOKURIKU, \ KYUSHU & :\textbf{「心不全パンデミック」}\\
SHIKOKU &:\textbf{「腰痛」} \\
\end{tabular}

\section{応募締め切りとプログラム期間}
\begin{tabular}{cl}
応募締め切り &:5/31(火)23:59 \\
二次選考期間 &:6/12(日)~ \\
プログラム期間 &:7/10(日)~ \\
\end{tabular}

\section{対象者}
中高生に相当する生徒または学生。(詳細は、注意事項に記載しております。)

\section{応募方法}
2~4名からなるチーム毎に応募フォームを提出。

\section{新型コロナウイルス対策について}
\begin{itemize}
\item 10月上旬までの教育プログラムやメンタリングは、オンライン化を積極的に進めています。
YouTubeなどによるオンラインコンテンツの配信も計画しております。
\item オフラインや対面で実施する場合も、自治体や政府・会場の指示に従い、マスクの着用、検温、手指消毒、換気など、感染症対策に万全を期して実施します。\\
なお、各地域の最終選考会と最終フォーラムは、オフライン・対面で行う予定です。\\ 
\end{itemize}
\rightline{\textreferencemark この情報は2022年4月7日現在のものであり、今後の情勢によっては変更の可能性もございます。}

\clearpage

\part*{参加に関する規約} 
プログラム中に以下の行動が見られた場合は、予告なく、参加をお断りする場合があります。
\begin{enumerate}[\ \ \bf 1. \ ]
\item 他の方に迷惑になる行為
\item 自己の利益のみを優先する行為
\item 消極的な活動
\item 一方的な否定や誹謗中傷など、仲間を傷つける行為
\item 過度な宣伝やスパム、無関係なリンクなどの投稿
\item 運営の許可なく、内部のコンテンツを転載したり、悪用する行為
\item 虚偽の回答やコミュニケーション
\end{enumerate}
\vspace{3mm}

\part*{応募と参加における注意事項}
\setcounter{section}{0}
\renewcommand{\thesection}{\Roman{section}.}

\section{エントリーする地域について}
{\bf チーム代表者の在籍学校が所在する都道府県}から、地域を選んでください。以下の表を参考にしてください。
\vspace{-5pt} \par
\begin{table}[h]
 \caption*{\bf i-GIP 地域 都道府県 割り振り}
 \centering
\begin{tabular}{rcll}
\hline\hline
&〈KANSAI〉	& 愛知県、三重県、滋賀県、京都府、大阪府、兵庫県、&\\
&			& 奈良県、和歌山県、鳥取県、島根県、岡山県&\\
\hline
&〈KANTO〉	& 北海道、青森県、岩手県、宮城県、秋田県、山形県、&\\
&			& 福島県、茨城県、栃木県、群馬県、埼玉県、千葉県、&\\
&			& 東京都、神奈川県、山梨県、長野県、静岡県 &\\
		\hline
&〈HOKURIKU〉& 新潟県、富山県、石川県、福井県、岐阜県 &\\
\hline
&〈SHIKOKU〉	& 徳島県、香川県、愛媛県、高知県&\\
\hline
&〈KYUSHU〉	& 広島県、山口県、福岡県、佐賀県、長崎県、熊本県、&\\
&			& 大分県、宮崎県、鹿児島県、沖縄県&\\
\hline\hline
\end{tabular}
\end{table}
\vspace{-5pt} \par

\section{チームを構成するメンバーについて}
\begin{enumerate}
\item チームメンバーの重複は認めません。\par
もしあった場合は、すべて{\bf 不合格 }とさせていただきます。
\item チームを構成するメンバーは、同じ学校や学年である{ \bf 必要はありません。}
\item メンバー全員が、以下に該当する中高生に相当する生徒または学生としてください。\par
ここにない学校種別につきましては、ご相談ください。
\begin{itemize}
\item 中学校・高等学校{\small(専攻科除く)}・中等教育学校・海上技術学校・高等専修学校{\small(専門学校の高等課程)}の全生徒
\item 高等専門学校に在学する1年生から3年生の学生
\item 義務教育学校に在学する7年生から9年生の生徒
\item 特別支援学校の中等部・高等部に在学する生徒
\end{itemize}
\end{enumerate}


\section{保護者様について}
応募に際しては、保護者様に一声かけていただくことを推奨します。

\section{写真や名前・在学情報について}
\begin{enumerate}
\item 応募フォームにおいて、名前や在学情報・写真使用に関する許諾をとります。(選考に影響しません。)\\
	「写真を非公開にしてほしい」や「在学情報を伏せてほしい」など、遠慮なく記入ください。
\item i-GIPプログラム中の写真は、運営広報より、共有しますので、個人のSNSで活用ください。
\end{enumerate}

\section{応募フォームの回答について}
\begin{enumerate}
\item 状況により応募フォームの回答が保存されないことがあります。十分にご注意ください。
\item 応募フォームを送付しますと、確認メールが送られます。もし、メールを確認できない場合は、再度応募フォームを提出してください。
\item 応募フォームの回答内容・提出内容をもとに一次選考を実施します。
\end{enumerate}

\section{連絡手段について}
詳細は応募フォーム上に記載しておりますが、メール{(\small info@inochi-gakusei.com)}や公式LINEなどの運営からの連絡を、{\bf 確実に}取れるようにしてください。

\section{制作物について}
\begin{enumerate}
\item i-GIPでの制作物のすべての知的財産権は、参加者本人に属します。
\item i-GIPのアイデアを、他の小論文・ビジネスコンテスト等に投稿することや発表することは禁止しません。
\begin{itemize}
\item[ ] ただし、投稿するコンテストの規定には従ってください。\par
\item[ ] 特に、多重応募が禁止されているコンテストでは、ご遠慮いただくなど、i-GIPを優先することを約束してください。
\end{itemize}
\item i-GIPに関する内容をブログやSNSなどに書いたり、解説を載せる場合は、その公開される情報の範囲に注意してください。不安な場合は、運営やメンターに聞いてください。
\begin{itemize}
\item[ ] 公開状況によっては、将来的に、特許権や実用新案権を取得しようとする際に、新規性を失う可能性があります。
\end{itemize}
\item i-GIPでの制作物を自身の責任でGithubにUPすることは止めません。
\end{enumerate}

\section{お金について}
\begin{enumerate}
\item 本プログラムはすべて、無料で行います。
\item すべてのプログラムにおいて、交通費と宿泊費は出せません。
\item その他のかかる必要{\small (プロトタイプ制作費など)}は各地域の指示に従ってください。
\end{enumerate}

\vspace{5mm}

\part*{個人情報に関する確認事項}
i-GIPでは,別記のプライバシーポリシーに従い、個人情報を取り扱います。\par
今回、応募フォームで収集した個人情報は、地域ごとに以下の担当者が統括管理し、限られたメンバーしか閲覧できないようにしております。また、以下の用途で使用いたします。\\

\vspace{-6pt}
\noindent 担当者(所属は2022年4月7日現在)\par \vspace{6pt}
\begin{tabular}{lcl}
i-GIP KANSAI & 西川 采那 & (大阪大学 医学部 医学科 2年)\\
i-GIP KANTO & 山下 彩夏 & (東京大学 文科一類 2年)\\
i-GIP HOKURIKU &谷口 千尋 & (金沢大学 医薬保健学域 医学類 2年)\\
i-GIP SHIKOKU & 大西 健斗 & (徳島大学 医学部 医学科 1年)\\
i-GIP KYUSHU & 吉田 尭史 & (久留米工業高等専門学校 専攻科 2年)\\
\end{tabular}
\vspace{12pt}
\\
用途
\begin{itemize}
\vspace{-9pt}
\item[$\bullet$] i-GIP 2022に関するご連絡と運営
\item[$\bullet$] inochi未来プロジェクトが主催するイベントのご案内
\end{itemize}

\vspace{15mm}

\rightline{\large 以上、2022年4月7日 \hspace{8mm}}
\vspace{1cm}
\begin{tabular}{ll}
{\bf \large 要項案作成} 		&  \\
\ \ \ \normalsize  i-GIP KYUSHU 所属	&吉田 尭史 \\
					& (久留米工業高等専門学校 専攻科 2年)\\
			\vspace{-9pt} \\
{\bf \large 修正・承認} & \\
\ \ \ inochi WAKAZO Project 全体代表 	& 北野 幸一郎\\
									&(京都府立医科大学 医学部 医学科 2年)\\
				\vspace{-6pt} \\
\ \ \ i-GIP 全地域統括				&島 碧斗  \\
									&(東京大学 文科一類 2年)
\end{tabular}
\end{document}
